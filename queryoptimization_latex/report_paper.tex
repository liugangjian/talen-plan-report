\documentclass[UTF8]{ctexart}
%%%%%%%%%%%%%%%%%%%%%%%%%%%== 引入宏 ==%%%%%%%%%%%%%%%%%%%%%%%%%%%%%
\usepackage{cite}
\usepackage{amsmath}	% 使用数学公式
\usepackage{graphicx}	% 插入图片/PDF/EPS 等图像
\usepackage{subfigure}	% 使用子图像或者子表格
\usepackage{geometry}	% 设置页边距
\usepackage{fancyhdr}	% 设置页眉页脚
\usepackage{setspace}	% 设置行间距
\usepackage{hyperref}	% 让生成的文章目录有链接,点击时会自动跳转到该章节
\usepackage{url}
\usepackage{caption2}
\usepackage{forest}
\usepackage{float}
\usepackage{hyperref}

\def\ojoin{\setbox0=\hbox{$\bowtie$}%
  \rule[-.02ex]{.25em}{.4pt}\llap{\rule[\ht0]{.25em}{.4pt}}}
\def\leftouterjoin{\mathbin{\ojoin\mkern-5.8mu\bowtie}}
\def\rightouterjoin{\mathbin{\bowtie\mkern-5.8mu\ojoin}}
\def\fullouterjoin{\mathbin{\ojoin\mkern-5.8mu\bowtie\mkern-5.8mu\ojoin}}

%%%%%%%%%%%%%%%%%%%%%%%%%%== 设置全局环境 ==%%%%%%%%%%%%%%%%%%%%%%%%%%%%
% [geometry] 设置页边距
\geometry{top=2.6cm, bottom=2.6cm, left=2.45cm, right=2.45cm, headsep=0.4cm, foot=1.12cm}
% 设置行间距为 1.5 倍行距
\onehalfspacing
% 设置页眉页脚
\pagestyle{fancy}
%\lhead{左头标}
%\chead{\today}
%\rhead{152xxxxxxxx}
\lfoot{}
\cfoot{\thepage}
\rfoot{}
%\renewcommand{\headrulewidth}{0.4pt}
%\renewcommand{\headwidth}{\textwidth}
%\renewcommand{\footrulewidth}{0pt}

%%%%%%%%%%%%%%%%%%%%%%%%%%== 自定义命令  ==%%%%%%%%%%%%%%%%%%%%%%%%%%%%%%
% 此行使文献引用以上标形式显示
\newcommand{\supercite}[1]{\textsuperscript{\cite{#1}}}
% 此行使section中的图、表、公式编号以A-B的形式显示
\renewcommand{\thetable}{\arabic{section}-\arabic{table}}
\renewcommand{\thefigure}{\arabic{section}-\arabic{figure}}
\renewcommand{\theequation}{\arabic{section}-\arabic{equation}}
% 此行使图注、表注与编号之间的分隔符缺省,默认是冒号:
\renewcommand{\captionlabeldelim}{~}

%===================================== 标题设置  ==========================================
% \heiti \kaishu 为字体设置,ctex 会自动根据操作系统加载字体
\title{\huge{\heiti Talent-Plan Section 3}}
\author{\small{\kaishu 宋阳}\\[2pt]
\small{\kaishu 北京航空航天大学}\\[2pt]
\small{Email:}
\url{yangsoonlx@gmail.com}
}
\date{} % 去除默认日期
%\date{\today}

%===================================== 正文区域  ==========================================
\begin{document}
\maketitle
% \tableofcontents % 目录内容,注释取消掉可实现目录

\begin{flushleft}
\textbf{课程目标}:熟悉数据库基础知识 \\[8pt]
\end{flushleft}
\section{课程作业}\label{sec1}

select $t_1.a$, count(*), avg($t_1.b$) from $t_1$ left outer join $t_2$ on $t_1.a=t_2.a$ group by $t_1.a$,请给出
所有可能的逻辑执行计划(画出Plan树),并分析$t_1$的数据分布对各种逻辑执行计划执行性能的影响。
查询的初步逻辑查询计划如下图所示: \\
\begin{figure}[H] 
  \begin{center}
    \fontsize{12pt}{12pt}\selectfont
    \begin{forest}
        [, phantom, s sep = 1cm
            [$\gamma_{t_1.a,\,count(*),\,avg(t_1.b)}$
              [$\mathop{\leftouterjoin}\limits_{t_1.a\,=\,t_2.a}$  
                  [$t_1$
                  ]
                  [$t_2$]
              ]
            ]
        ]
    \end{forest}
  \end{center}
  \caption{初步逻辑查询图} \label{tree1}
\end{figure}
\section{执行计划分析}\label{sec2}
查询编译可以分为3个步骤: \textbf{1.建立查询的分析树 2.查询重写 3.物理计划生成}。本文的行文思路是先考虑不同表属性情况下,对初始的分析树\ref{tree1}
进行查询重写,然后针对各个查询树进行物理计划生成分析。

如果优化器想判断哪个查询计划执行最快,就必须估计代价,在本文我们使用了一些描述操作符代价的参数。对于一个关系$R$,\textbf{在本文我们假设关系$R$是聚集的}:

1. $B(R)$表示关系$R$所需磁盘块的个数,$M$表示关系$R$可以获取到的内存块的个数。

2. $T(R)$表示关系$R$中元组的个数。

3. $V(R, a)$表示关系$R$中属性$a$中不同值的个数。

对于sql的查询分析树中的一些关系操作,我们使用操作符号进行代替,如: $\gamma$(分组操作符),$\pi$(投影操作符),$\delta$(消除重复)
$\leftouterjoin$(左外连接)。

逻辑优化主要是基于规则的优化,数据库逻辑优化规则包括:列裁剪,最大最小消除,投影消除,谓词下推等等。
本次课程作业的sql语句包括连接和聚合操作,针对这类sql语句可以使用聚合消除,外连接消除等逻辑优化规则。
逻辑优化会针对sql语句中算子的不同性质进行不同的优化操作。所以接下来开始讨论在下面这几种不同的组合下可以进行的
逻辑优化操作。

\begin{enumerate}
	\item 属性$t_{1}.a$在表$t_{1}$中不具有唯一性,属性$t_{2}.a$在表$t_{2}$中也不具有唯一性;
	\item 属性$t_{1}.a$在表$t_{1}$中具有唯一性,属性$t_{2}.a$在表$t_{2}$中不具有唯一性;
	\item 属性$t_{1}.a$在表$t_{1}$中不具有唯一性,属性$t_{2}.a$在表$t_{2}$中具有唯一性;
	\item 属性$t_{1}.a$在表$t_{1}$中具有唯一性,属性$t_{2}.a$在表$t_{2}$中具有唯一性;
\end{enumerate}

\subsection{属性$t_{1}.a$和属性$t_{2}.a$均不具有唯一性}
\textbf{逻辑执行优化}: 
第一种组合是最普通的组合,即属性$t_{1}.a$在表$t_{1}$中不具有唯一性,属性$t_{2}.a$在表$t_{2}$中也不具有唯一性。这时候,只能对图\ref{tree1}中的最初始的执行计划进行优化:
从sql语句中可以看到,最终的输出结果只涉及到了表$t_1$中的属性a和属性b,表$t_2$只是用来做连接操作的。所以我们使用列裁剪优化规则,裁剪掉用不上的列。优化后的逻辑查询计划如图\ref{tree2}所示:

\begin{figure}[H] 
  \begin{center}
    \fontsize{15pt}{15pt}\selectfont
    \begin{forest}
        [, phantom, s sep = 1cm
            [$\gamma_{t_1.a,\,count(*),\,avg(t_1.b)}$
              [$\mathop{\leftouterjoin}\limits_{t_1.a\,=\,t_2.a}$  
                  [$\pi_{t_1.a, \, t_1.b}$
                    [$t_1$]
                  ]
                  [$\pi_{t_2.a}$
                    [$t_2$]
                  ]
              ]
            ]
        ]
    \end{forest}
  \end{center}
  \caption{属性$t_{1}.a$和属性$t_{2}.a$均不具有唯一性时的逻辑查询图} \label{tree2}
\end{figure}

\textbf{物理执行分析}: 

1. 通过逻辑执行优化,我们通过列裁剪的方式,只选用上层算子需要的属性,投影不能减少$T(t_1)$值,但是
可以使得读取的元组大小被有效的缩减,为了上层的Left Outer Join减少内存需求。

2.当执行到Left Outer Join时:

(1)当$B(t_2)<M$,我们可以使用一趟算法,将关系$t_1$放到$M-1$的内存块中,依据$t_2.a$构造适当的
查询结构,将$t_1$按块读入剩下的块中,用$t_1.a$在$M-1$块中查找匹配的值。这个过程需要$B(t_1)+B(t_2)$次
磁盘IO。

(2)当$B(t_2)>M$, 我们可以使用Hash Join,Sort Merge Join,Nested-Loop Join。这里主要讨论
Hash Join和Sort Merge Join。

\textbf{Hash Join}
因为使用的是Left Outer Join,我们只能使用$t_2$作为Inner表,在上面构建哈希表。
左表作为驱动表进行哈希匹配,Hash Join适合在Join连接项不是索引的情况下。大约需要$3*(B(t_1)+B(t_2))$次
磁盘IO。

\textbf{Sort Merge Join} 

\textbf{当$t_1.a$和$t_2.a$均不是索引的时候},分为2阶段:第一阶段,我们将关系$t_1$和$t_2$分别划分为
$B(t_1)/M$和$B(t_2)/M$个子表,在$B(R)+B(S) \leq M^2$的前提下,对每个子表进行排序,存储到外存上; 第二阶段,把每个
子表加载到内存中,因为子表总个数小于$M$,我们可以在子表上进行归并排序进行匹配连接,最后输出结果。大约需要$3*(B(t_1)+B(t_2))$次
磁盘IO。

\textbf{当考虑到索引,就涉及到基于索引的连接。}

\textbf{当$t_1.a$和$t_2.a$均是索引的时候},就不需要第一阶段的排序过程,可以直接从内存中读取索引,直接比较两个关系的索引值,
但因为是Left Outer Join,左表可能需要去磁盘读取所有的值,当匹配值相等时左表去磁盘取出相应的值;也可能不需要进行磁盘IO,
因为如果$t_1.a$是二级索引,$t1.b$是聚簇索引,左表不需要进行磁盘读写,直接从二级索引中就能返回$t1.b$的值。
直接内存操作就可以,若不能把索引都加载到内存中,也需要极少的磁盘IO。

\textbf{当$t_1.a$和$t_2.a$中有一个是索引的时候},这种情况和上面类似,可以避免进行对关系的排序,减少磁盘IO。

3. 分组聚合

\textbf{基于一趟算法的分组聚合}: 对于中间聚合结果$\mathop{\leftouterjoin}\limits_{t_1.a\,=\,t_2.a}$,如果$B(\delta(\mathop{\leftouterjoin}\limits_{t_1.a\,=\,t_2.a}))<M$,
表示我们可以把每一种$t_1.a$的项存在内存中用来统计每个分组中的$sum(t_1.b)$和$count(*)$,然后留出一块内存,用来
遍历关系$\mathop{\leftouterjoin}\limits_{t_1.a\,=\,t_2.a}$,统计结果,直到扫描到最后一个块时,才输出结果。
需要$B(\mathop{\leftouterjoin}\limits_{t_1.a\,=\,t_2.a})$磁盘IO。

\textbf{基于hash的分组聚合}: 我们先将$\mathop{\leftouterjoin}\limits_{t_1.a\,=\,t_2.a}$的所有元组散列到$M-1$个桶里,
只要满足$B(\mathop{\leftouterjoin}\limits_{t_1.a\,=\,t_2.a})<M^2$, 就可以在内存中处理每一个桶,计算每个桶中的分组聚合数目,
并输出结果,需要$3*B(\mathop{\leftouterjoin}\limits_{t_1.a\,=\,t_2.a})$次的磁盘IO。

\textbf{基于排序的分组聚合}: 如果之前进行Left Outer Join时候选择的是Sort Merge Join,那么$\mathop{\leftouterjoin}\limits_{t_1.a\,=\,t_2.a}$就是按$t_1.a$顺序存储,
可以将$B(\mathop{\leftouterjoin}\limits_{t_1.a\,=\,t_2.a})$划分为$B(\mathop{\leftouterjoin}\limits_{t_1.a\,=\,t_2.a})/M$块,每次读入子块,检测每个排序关键字为$v$
的元组,进行累计所需聚集,如果内存块空了,就继续读下一个子块。只需要$B(\mathop{\leftouterjoin}\limits_{t_1.a\,=\,t_2.a})$次的
磁盘IO。

\subsection{属性$t_{1}.a$具有唯一性,属性$t_{2}.a$不具有唯一性}
\begin{figure}[H] 
  \begin{center}
    \fontsize{15pt}{15pt}\selectfont
    \begin{forest}
        [, phantom, s sep = 1cm
            [$\gamma_{t_1.a,\,count(*),\,avg(t_1.b)}$
              [$\mathop{\leftouterjoin}\limits_{t_1.a\,=\,t_2.a}$  
                  [$\pi_{t_1.a, \, t_1.b}$
                    [$t_1$]
                  ]
                  [$\pi_{t_2.a}$
                    [$t_2$]
                  ]
              ]
            ]
        ]
    \end{forest}
  \end{center}
  \caption{属性$t_{1}.a$具有唯一性,属性$t_{2}.a$不具有唯一性时的逻辑查询图} \label{tree3}
\end{figure}

\textbf{逻辑执行优化}: 这种情况和2.1的情况一样,逻辑执行计划和图\ref{tree2}一致,但是有一点不同之处,因为$t_1.a$的属性是唯一性的,
所以对应的$t_1.b$也是唯一的,其实数据库在进行avg运算的时候,可以直接输出$t1.b$。

\textbf{物理执行分析}: 关于物理执行分析和2.1的也一致,在此就不再赘述。


\subsection{属性$t_{1}.a$不具有唯一性,属性$t_{2}.a$具有唯一性}
\textbf{逻辑执行优化}: 当$t_{2}.a$具有唯一性属性时,sql语句中的左外连接算子可以进行消除。
对于表$t_{1}$的每一行来说,经过左外连接之后,最终只产生一行连接结果,
而且,最终输出的结果只需要表$t_1$中的属性。因此,优化后的逻辑查询结果如图\ref{tree4}
所示:
\begin{figure}[H] 
  \begin{center}
    \fontsize{15pt}{15pt}\selectfont
    \begin{forest}
        [, phantom, s sep = 1cm
            [$\gamma_{t_1.a,\,count(*),\,avg(t_1.b)}$
              [$\pi_{t_1.a,\,1,\,t_1.b}$
                  [$t_1$]
              ]
            ]
        ]
    \end{forest}
  \end{center}
  \caption{属性$t_{1}.a$不具有唯一性,属性$t_{2}.a$具有唯一性时的逻辑查询图} \label{tree4}
\end{figure}

\textbf{物理执行分析}: 在本节条件下,我们考虑选择一个合适的分组聚合算法:

\textbf{基于一趟算法的分组聚合}:如果$B(\delta(t_1))<M$,
表示我们可以把每一种$t_1.a$的项存在内存中用来统计每个分组中的$sum(t_1.b)$和$count(*)$,然后留出一块内存,用来
遍历关系$t_1.a$,统计结果,直到扫描到最后一个块时,才输出结果,需要$B(t_1)$次磁盘IO。

\textbf{基于hash的分组聚合}: 如果$B(\delta(t_1))>M$, 我们先将$t_1$的所有元组散列到$M-1$个桶里,
只要满足$B(t_1)<M^2$, 就可以在内存中处理每一个桶,计算每个桶中的分组聚合数目,并输出结果,
需要$3*B(t_1)$次的磁盘IO,基于hash的分组聚合,适合在$t_1.a$非索引的情况下。

\textbf{基于排序的分组聚合}: 

当$B(\delta(t_1))>M$,\textbf{$t_1.a$不是索引的时候},同样分为2个阶段:首先,我们将$t_1$划分为
$B(t_1)/M$个子块,分别对每个子块进行排序;第二阶段,将每个子表的第一块加载到内存中,然后查找第一个最小的值$v$,并使用$v$
进行聚集计算。当某一子块的内存块读取完毕就读取下一块到内存,需要$3*B(t_1)$次磁盘IO,而且需要保证$B(t_1)<M^2$。

当$B(\delta(t_1))>M$,\textbf{$t_1.a$是索引的时候},如果$t_1.a$是聚簇索引,所以$t_1$是按照$t_1.a$顺序排序,
之后的处理和2.1的\textbf{基于排序的分组聚合}处理类似。如果$t_1.a$是二级索引,$t_1.b$是聚簇索引,就需要很少的磁盘IO,
在内存中处理就可以。

\subsection{属性$t_{1}.a$和属性$t_{2}.a$均具有唯一性}
\textbf{逻辑执行优化}:首先根据2.3的结果,我们可以做外连接消除优化。又因为$t_1$也具有唯一性,所以可以继续做聚集消除操作。
因此,优化后的逻辑查询结果如图\ref{tree5}所示:
\begin{figure}[H] 
  \begin{center}
    \fontsize{15pt}{15pt}\selectfont
    \begin{forest}
        [, phantom, s sep = 1cm
            [$\pi_{t_1.a,\,1,\,t_1.b}$
                [$t_1$]
            ]
        ]
    \end{forest}
  \end{center}
  \caption{属性$t_{1}.a$和属性$t_{2}.a$均具有唯一性时的逻辑查询图} \label{tree5}
\end{figure}

\textbf{物理执行分析}: 在这种情况下,如果在没有索引的情况下,就是简单的做扫描表关系就行。
如果$t_1.a$是二级索引,$t_1.b$是聚簇索引。就可以直接在内存处理,减少磁盘IO。

\section{参考文献}


\begin{enumerate}
	\item \href{https://book.douban.com/subject/4838430/}{数据库系统实现} 
	\item \href{https://book.douban.com/subject/25815707/}{数据库查询优化器的艺术} 
\end{enumerate}

\end{document}